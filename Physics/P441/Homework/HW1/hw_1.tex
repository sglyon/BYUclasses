%% Preamble
  \documentclass{homework}

  \hwTitle{Assignment\ \#1 - Math Review} % Assignment title
  \hwDueDate{Friday,\ May\ 10,\ 2013} % Due date
  \hwClass{Physics\ 441} % Course/class
  \hwInstructor{Berrondo} % Teacher/lecturer
  \hwAuthor{Spencer Lyon} % Your name

% New commands I use a lot
  \newcommand{\bs}[1]{\ensuremath{\boldsymbol{#1}}}
  \newcommand{\cross}[2]{\ensuremath{\boldsymbol{#1} \times \boldsymbol{#2}}}

  % partial derivative as a fraction
  \newcommand{\fracpd}[2]{
    \ensuremath{\frac{\partial #1}{\partial #2}}
  }

  % Inline curl definition
  \newcommand{\smcurl}{
  \ensuremath{\left|
        \begin{smallmatrix}
        \bs{\hat{x}} & \bs{\hat{y}} & \bs{\hat{z}} \\
        \partial / \partial x & \partial / \partial y & \partial / \partial z \\
        v_x & v_y & v_z
        \end{smallmatrix}
        \right|}
  }

  % inline curl of vector
  \newcommand{\smcurldet}[3]{
  \ensuremath{\left|
        \begin{smallmatrix}
        \bs{\hat{x}} & \bs{\hat{y}} & \bs{\hat{z}} \\
        \partial / \partial x & \partial / \partial y & \partial / \partial z \\
        #1 & #2 & #3
        \end{smallmatrix}
        \right|}
  }

  % expand curl determinant
  \newcommand{\expandcurl}[3]{
   \ensuremath{
   \bs{\hat{x}} \left(\fracpd{(#3)}{y} - \fracpd{(#2)}{z} \right) - \bs{\hat{y}} \left(\fracpd{(#3)}{x} - \fracpd{(#1)}{z} \right) + \bs{\hat{z}} \left(\fracpd{(#2)}{x} - \fracpd{(#1)}{y} \right)
   }
   }

   % Just a vector in xhat yhat zhat
   \newcommand{\xyzvec}[3]{
   \ensuremath{
      (#1) \bs{\hat{x}} + (#2) \bs{\hat{y}} + (#3) \bs{\hat{z}}
   }
   }

\begin{document}

\maketitle

% Problem 1
  \begin{homeworkProblem}[Problem 1.2]
      Is the cross product associative? $$ (\cross{A}{B}) \times \bs{C} \stackrel{?}{=} \bs{A} \times (\cross{B}{C})$$ If so prove it, if not provide a counter example
      \vspace{.2in}

      \problemAnswer{ % Answer
        Let $\bs{A} = \bs{i} = <1, 0, 0>$, $\bs{B} = \bs{i} = <1, 0, 0>$, $\bs{C} = \bs{k} = <0, 0, 1>$.

        In this case $$ (\cross{A}{B}) \times \bs{C} =  (<0, 0, 0>) \times <0, 0, 1> = <0, 0, 0>$$

        and $$ \bs{A} \times (\cross{B}{C}) = <1, 0, 0> \times (<0, -1, 0>) = <0, 0, -1>$$

        This two aren't equal to the cross product is non associative. \qed
      }
  \end{homeworkProblem}

% Problem 2
  \begin{homeworkProblem}[Problem 1.4]
    Use the cross product to find the components of the unit vector $\hat{n}$ that is normal to the plane passing through the following points:  $<1, 0 0>, <0, 2, 0>, <0, 0, 3>$.
    \vspace{.2in}
    \problemAnswer{ % Answer
       We need to find two vectors parallel to the plane. We can use $<1, -2, 0>$ and $<0, 2, -3>$. Now that we have those vectors we can take the cross product between them and normalize the resultant vector. $$<1, -2, 0> \times <0, 2, -3> = <6, 3, 2>$$

     Now we normalize this vector to make it a unit vector: $$ \frac{<6, 3, 2>}{\text{norm}(<6, 3, 2>)} = \frac{<6, 3, 2>}{7} = <0.86,  0.43,  0.29>$$ \qed
    }
  \end{homeworkProblem}

% Problem 3
  \begin{homeworkProblem}[Problem 1.9]
    Find the transformation matrix $R$ that describes rotation by $120^{\circ}$ about an axis from the origin through the point $(1, 1, 1)$. The rotation is clockwise as you look down the axis toward the origin.
    \vspace{.2in}
    \problemAnswer{ % Answer
      The general form for the rotation matrix of a vector about an axis of direction $k$ is given below\footnote{This algorithm was found at \url{http://en.wikipedia.org/wiki/Euler\%E2\%80\%93Rodrigues_parameters}}:
      $$R = \begin{pmatrix}
                  a^2+b^2-c^2-d^2 & 2(bc-ad) & 2(bd + ac) \\
                  2(bc+ad) & a^2+c^2-b^2-d^2 & 2(cd - ab) \\
                  2(bd-ac) & 2(cd+ab) & a^2+d^2-b^2-c^2
                  \end{pmatrix}$$

      where $\bs{k} = (k_x, k_y, k_z)$ is the axis of rotation and

      \begin{align*}
      a & = \cos (\phi/2); \\
      b & = k_x \sin (\phi/2); \\
      c & = k_y \sin (\phi/2); \\
      d & = k_z \sin (\phi/2).
      \end{align*}

      In our case $\phi = 120^{\circ}$ and $\bs{k} = <1, 1, 1>$. Plugging those things in that $a = 0.5$ and $b=c=d=0.866$. Plugging these values in to the matrix above we get:

      $$ R = \begin{bmatrix}
       0 & 0 &1 \\
       1 & 0 & 0 \\
       0 & 1 & 0
      \end{bmatrix} $$ \qed
    }
  \end{homeworkProblem}

% Problem 4
  \begin{homeworkProblem}[Problem 1.12]
    The height of a certain hill (in feet) is given by $$ h(x, y) = 10(2xy - 3x^2 - 4y^2 0 18x + 28y + 12)$$ where $y$ is the distance (in miles) north, $x$ is the distance east of South Hadley.

    \begin{enumerate}
      \item Where is the top of the hill located?
      \item How high is the hill?
      \item How steep is the slope (in feet per mile) at a point 1 mile north and one mile east of South Hadley? In what direction is the slope steepest, at that point?
    \end{enumerate}

    \problemAnswer{ % Answer
      To answer these questions we need the gradient $$\nabla h(x,y) = 10\left((2y - 6x - 18) \hat{x} + (2x - 8y + 28) \hat{y}\right)$$
      \begin{enumerate}
        \item The top of the hill is where $\nabla h(x, y) = 0$. To find this we set each component equal to $0$ and solve the linear system.
        \begin{align*}
          0 &= 10\left((2y - 6x - 18) \hat{x}\right) \\
            & = 2y - 6x - 18 \\
            18 & = 2y - 6x
        \end{align*}

        \begin{align*}
          0 &= 10\left(  (2x - 8y + 28) \hat{y} \right) \\
            &= 2x - 8y + 28 \\
            -28 &= 2x - 8y
        \end{align*}

        We can now solve the equation $Ax=b$, where $A = \left(\begin{smallmatrix}  -6 & 2 \\ 2 & -8 \end{smallmatrix}\right)$,  $x = \left(\begin{smallmatrix} x \\ y \end{smallmatrix}\right)$, and $b = \left(\begin{smallmatrix} 18 \\ -28 \end{smallmatrix}\right)$

        This yields the solution that $x = -2, y = 3$, which means that the peak is 3 miles north and two miles west of South Hadley.

        \item For this part we just plug $(-2, 3)$ in to $h(x, y)$ and get $h^*(x, y) = h(-2, 3)  = 720$ ft.
        \item One mile north and one mile east of South Hadley means that $x = y = 1$. Plugging those into the gradient we get $$\nabla h(1, 1) = 10\left((2(1) - 6(1) - 18) \hat{x} + (2(1) - 8(1) + 28) \hat{y} \right) = 10 \left(-22 \hat{x} + 22 \hat{y} \right)$$

        The magnitude of this vector represents the steepness of the slope and is equal to $\sqrt{(-220)^2 + (220)^2} \approx 311.13$. Because the $x$ portion is negative and the $y$ portion is positive, the direction of steepest slope is northwest.
      \end{enumerate} \qed
    }
  \end{homeworkProblem}

% Problem 5
  \begin{homeworkProblem}[Problem 1.13]
    Let $\bs{\eta}$ be the separation vector from a fixed point $(x', y', z')$ to hte point $(x, y, z)$ and let $\eta$ be its length. Show that:

    \begin{enumerate}
      \item $\nabla(\eta^2) = 2 \bs{\eta}$
      \item $\nabla(1 / \eta) = -\bs{\eta} / \eta^2$
      \item What is the general formula for $\nabla(\eta^n)$?
    \end{enumerate}

    \vspace{.2in}
    \problemAnswer{ % Answer
      We begin by defining $\bs{\eta}$ $$\bs{\eta} = (x - x') \hat{x} + (y - y') \hat{y} + (z - z') \hat{z} $$

      \begin{enumerate}
        \item We recognize that $\eta^2$ is simply the sum of each squared component: $$\eta^2 = (x - x')^2 + (y - y')^2 + (z - z')^2$$ We can now compute $\nabla(\eta^2)$

          \begin{align*}
            \nabla(\eta^2) &= \fracpd{\eta^2}{x} \hat{x} + \fracpd{\eta^2}{y} \hat{y} + \fracpd{\eta^2}{z} \hat{z} \\
                  &= 2(x - x')\hat{x} + 2(y - y')\hat{y} + 2(z - z')\hat{z}\\
                  &= 2 \bs{\eta}
          \end{align*}

        \item We know that $\eta = \sqrt{\eta^2} = \left((x - x')^2 + (y - y')^2 + (z - z')^2\right)^{1/2} $. Using this we can say that $(1 / \eta) =  \left((x - x')^2 + (y - y')^2 + (z - z')^2\right)^{-1/2}$. We will use this definition to solve the problem.
        Note that I define $\xi \equiv \left((x - x')^2 + (y - y')^2 + (z - z')^2\right)^{-3/2} \equiv \eta^{1/3}$. Also note that $\bs{\eta} = \eta \hat{\bs{\eta}}$

          \begin{align*}
            \nabla(1 / \eta) &= \fracpd{(1 / \eta)}{x} \hat{x} + \fracpd{(1 / \eta)}{y} \hat{y} + \fracpd{(1 / \eta)}{z} \hat{z} \\
              &= \frac{-1}{2} \xi 2(x - x') \hat{x} - \frac{1}{2} \xi 2(y - y') \hat{y}- \frac{1}{2} \xi 2(z - z') \hat{z}\\
              &= -\xi \left[(x - x') \hat{x} + (y - y') \hat{y} + (z - z') \hat{z} \right] \\
              &= - \eta^{1/3} \bs{\eta} = -\hat{\bs{\eta}} / \eta^2
          \end{align*}

        \item The general formula is pretty easy:

          \begin{align*}
            \nabla (\eta^n) &= n \eta ^{n-1} \fracpd{\eta}{x} + n \eta ^{n-1} \fracpd{\eta}{y} + n \eta ^{n-1} \fracpd{\eta}{z}\\
              &= n \eta ^{n-1} \left[ \frac{1}{2} \eta^-1 2 \eta_x + \frac{1}{2} \eta^-1 2 \eta_y + \frac{1}{2} \eta^-1 2 \eta_z\right] \\
              &= n \eta ^{n-1} \left[ \hat{\bs{\eta_x}} + \hat{\bs{\eta_y}}+ \hat{\bs{\eta_z}} \right] \\
              &= n \eta ^{n-1}  \hat{\bs{\eta}}
          \end{align*}

      \end{enumerate} \qed
    }
  \end{homeworkProblem}

% Problem 6
  \begin{homeworkProblem}[Problem 1.16]
    Sketch the vector function $$\bs{v} = \frac{\bs{\hat{r}}}{r^2}$$ and compute its divergence. The ansewr may surprise you ... can you explain it?

    \begin{figure}[!h]
      \includegraphics[width=\linewidth, height=4in]{VectorField.png}
      \caption{The vector field $\bs{v} = \frac{\bs{\hat{r}}}{r^2}$}
      \label{fig:VectorField}
    \end{figure}

    \vspace{.2in}
    \problemAnswer{ % Answer

      The plot of the vector field can be found in Figure \ref{fig:VectorField}.

      For this problem $\bs{r} = x \hat{x} + y \hat{y} + z \hat{z}$ and  $r = \sqrt{x^2 + y^2 + z^2}$, so $\bs{v}$ becomes $$\bs{v} = \frac{\bs{\hat{r}}}{x^2 + y^2 + z^2} = \frac{\bs{r}}{(x^2 + y^2 + z^2) ^ {3/2}} = \frac{x \hat{x} + y \hat{y} + z \hat{z}}{(x^2 + y^2 + z^2) ^ {3/2}}$$

      The divergence can now be computed. Note that we make the substitution $\bs{\hat{r}} = \bs{r} / r$

      \begin{align*}
        \nabla \cdot \bs{v} &= \fracpd{\bs{v}}{x} + \fracpd{\bs{v}}{y} + \fracpd{\bs{v}}{z} \\
          &= \fracpd{(\bs{r} / r^3)}{x} + \fracpd{(\bs{r} / r^3)}{y} + \fracpd{(\bs{r} / r^3)}{z}\\
          &= \fracpd{}{x} \left[ x (x^2 + y^2 + z^2) ^{-3/2}\right] + \fracpd{}{y} \left[ y (x^2 + y^2 + z^2) ^{-3/2}\right] + \fracpd{}{z} \left[ z (x^2 + y^2 + z^2) ^{-3/2}\right] \\
          &=  3 (x^2 + y^2 + z^2) ^{-3/2} + (-3/2) (x^2 + y^2 + z^2) ^{-5/2} \left(2x^2 + 2y^2 + 2z^2 \right) \\
          &= 3 r^{-3}  - 3 r^{-5} \left( r^2\right)\\
          &= 3r^{-3} - 3r^{-3} = 0
      \end{align*}

      This is in the same form as Coulomb's law for a point charge. In this case the divergence represents the flux at a distance $r$. Because Coulomb's law describes point charges, the flux is zero for all radii not equal to 0. In other words, the only place flux exists is where the point charge is located. \qed
    }
  \end{homeworkProblem}

% Problem 7
  \begin{homeworkProblem}[Problem 1.18]
    Calculate the curl of the vector functions:
    \begin{enumerate}
      \item $\bs{v} = \xyzvec{x^2}{3xz^2}{-2xz}$
      \item $\bs{v} = \xyzvec{xy}{2yz}{3xz}$
      \item $\bs{v} = \xyzvec{y^2}{2xy + z^2}{2yz}$
    \end{enumerate}

    \vspace{.2in}
    \problemAnswer{ % Answer
      The curl of a vector $\bs{v}$ is defined as $$\nabla \times \bs{v} = \smcurl$$ We will apply this definition each vector function.
      \begin{enumerate}
        \item
        \begin{align*}
          \nabla \times \bs{v} &=  \smcurldet{x^2}{3xz^2}{-2xz} \\
            &= \expandcurl{x^2}{3xz^2}{-2xz} \\
            &= \xyzvec{-6xz}{2z}{3z^2}
        \end{align*}

        \item
        \begin{align*}
          \nabla \times \bs{v} &=  \smcurldet{xy}{2yz}{3xz} \\
            &= \expandcurl{xy}{2yz}{3xz}\\
            &= \xyzvec{-2y}{-3z}{-x}
        \end{align*}

        \item
        \begin{align*}
          \nabla \times \bs{v} &=  \smcurldet{y^2}{2xy + z^2}{2yz} \\
            &= \expandcurl{y^2}{2xy + z^2}{2yz}\\
            &= \xyzvec{2z - 2z}{0}{2y - 2y} \\
            &= 0
        \end{align*}

      \end{enumerate} \qed
    }
  \end{homeworkProblem}

% Problem 8
  \begin{homeworkProblem}[Problem 1.19]
    Draw a circle in the $xy$ plane. At a few representative points draw the vector $\bs{v}$ tangent to the circle, pointing in the clockwise direction. By comparing adjacent vectors, determine the sign of $\fracpd{v_x}{y}$ and $\fracpd{v_y}{x}$. According to the definition of the curl, what is the direction of $\nabla \times \bs{v}$? Explain how this example illustrates the geometrical interpretation of the curl.

    \begin{figure}[!h]
      \includegraphics[width=\linewidth, height=4in]{CurlField.png}
      \caption{Tangent vectors in the clockwise direction on a circle in the $xy$ plane.}
      \label{fig:CurlField}
    \end{figure}

    \vspace{.2in}
    \problemAnswer{ % Answer

    Moving from point 1 to point 2 in Figure \ref{fig:CurlField} we are increasing in $y$ and decreasing in $x$. We can also see that both $v_x$ and $v_y$ are increasing. Moving from point 3 to point 4 we see that we are  increasing in $x$ and decreasing in $y$, but that $v_x$ and $v_y$  are decreasing. From this analysis we can determine the following signs:

    \begin{align*}
      \fracpd{v_x}{y} &> 0 \\
      \fracpd{v_y}{x} &< 0
    \end{align*}

    Coupling this with the definition of the curl we see that $\nabla \times \bs{v}$ is in the negative $\bs{\hat{z}}$ direction (terms in the $\bs{\hat{x}}$ and $\bs{\hat{y}}$ directions are all zero). It follows the direction of the right hand rule. If we take our right hand and twist in the direction of the tangent vectors, we will have our thumb pointing directly into the circle, which is the $-\bs{\hat{z}}$ direction. \qed

    }
  \end{homeworkProblem}

  \end{document}
