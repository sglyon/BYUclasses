% Created 2013-05-28 Tue 11:20
\documentclass[11pt]{article}
\usepackage[utf8]{inputenc}
\usepackage[T1]{fontenc}
\usepackage{fixltx2e}
\usepackage{graphicx}
\usepackage{longtable}
\usepackage{float}
\usepackage{wrapfig}
\usepackage{soul}
\usepackage{textcomp}
\usepackage{marvosym}
\usepackage{wasysym}
\usepackage{latexsym}
\usepackage{amssymb}
\usepackage{hyperref}
\usepackage[mathscr]{euscript}
\usepackage{amsmath,bm}
\tolerance=1000
\providecommand{\alert}[1]{\textbf{#1}}

\title{Physics 441}
\author{Spencer Lyon}
\date{2013-05-01 Wed}
\hypersetup{
  pdfkeywords={notes, physics, EM},
  pdfsubject={},
  pdfcreator={Emacs Org-mode version 7.8.11}}

\begin{document}

\maketitle

\setcounter{tocdepth}{3}
\tableofcontents
\vspace*{1cm}


\section{Course Info}
\label{sec-1}

The slides are available online, but they are password protected. The password
is m@xwell.

TA Help sessions will be Thursday at Noon in N337

The \href{http://www.physics.byu.edu/faculty/berrondo/sp441/}{Class website} has good info.

The \href{http://www.physics.byu.edu/faculty/berrondo/sp441/homework12.html}{homework schedule} is here.
\section{Unit 1}
\label{sec-2}
\subsection{Introduction}
\label{sec-2-1}

Dr. Berrando likes to use Clifford Algebras to solve these problems. Our book
doesn't so in order for us to use them we need to be in class. He thinks they
make this class easier, but did say that people either hate them or love them.

We will be studying Electricity and Magnetism as a single field. Maxwell has us
think about vector fields and sources. His equations all take the form $\nabla
\dots = \dots$, where the dots on the rhs stands for $\cdot$ or $\times$ some
field. The dots on the right stand for a source. In this class they will all be
static (time independent).

As an example of these principles and what things look like in a Clifford
Algebra we would write: $$\nabla \mathscr{F} = \tilde{J}$$ Clifford Algebras
make solving this for $\mathscr{F}$ very easy: $$\mathscr{F}=\nabla^{-1}J$$
\subsection{Tools}
\label{sec-2-2}

\begin{table}[htb]
\caption{The rows of this table don't align. A table was just a compact way to show the data}
\begin{center}
\begin{tabular}{llll}
 Math                                               &  Physics                                 &     &     \\
\hline
 trigonometry                                       &  Trajectories r(t)                       &     &     \\
 vectors: dot, cross, Clifford                      &  Fields (scalar-vector, static-dynamic)  &     &     \\
 vector derivative operators ($\nabla$)             &  Sources (charge, current)               &     &     \\
 Dirac Delta function                               &  Superposition of soureces               &     &     \\
 Discrete to continuum                              &  Superposition of fields                 &     &     \\
 integral theorems (stokes, gauss -- inside cover)  &  unit point sources                      &     &     \\
 cylindrical and spherical coords                   &  maxwell's Equaitons                     &     &     \\
 linearity                                          &  field lines                             &     &     \\
                                                    &  charge conversation                     &     &     \\
                                                    &  potentials                              &     &     \\
\end{tabular}
\end{center}
\end{table}
\subsubsection{Math Review}
\label{sec-2-2-1}

\begin{itemize}
\item Sum of vectors
\item dilation (multiplication by scalar)
\item Linear combinations (put previous two points together)
\item Scalar (dot) Product: $\bm{A} \cdot \bm{B} = A B
  cos(\theta)$, where $A = \sqrt{\bm{A} \cdot \bm{A}}$
\item Cross product: $\bm{A} \times \bm{B} = \bm{n} A
  B |sin(\theta)|$
\item Orthonormalbasis: $\{e_1, e_2, e_3\} = \{i, j, k\}$
\item Triple dot (scalar) product: One cross and a dot. It is cyclically
  constant. i.e. $A \cdot (B \times C) = B \cdot (C \times A) =C \cdot (A
  \times B)$. It gives you the volume of the parallelipiped defined by the
  three vectors.
\item The triple vector product has two crosses. It is non-associative. Rule: $A
  \times (B \times C) = B(A \cdot C) - C(A \cdot B)$
\item Rotation of a vector in 3-d. A unit vector ($n$) defines the rotation axis
  and $\phi$ defines the rotation angle. We can express this as \$r' = e$^{\phi n
  \times}$ r = e$^{\phi n \times}$ (r$_{\parallel}$ + r$_{\perp}$) = r$_{\parallel}$ +
  e$^{\phi n \times}$ r$_{\perp}$ = r$_{\parallel}$ + cos($\phi$) r$_{\perp}$ + sin($\phi$)
  n \texttimes{} r\$
\begin{itemize}
\item Example: Rotate vector $e_1 + e_2$ by 45 degrees. Here $\phi = 45$, $r =
    (e_1 + e_2)$, $n = e_3$. Plugging it in we get $r' = (\frac{1}{\sqrt{2}} +
    \frac{1}{\sqrt{2}} e_3 \times) (e_1 + e_2)$
\item Algorithm for finding rotation
\begin{enumerate}
\item Make sure $\hat{n}$ is a unit vector
\item Find $r_{\parallel} = (\hat{n} \cdot r) \hat{n}$
\item Find $r_{\perp} = r - r_{\parallel}$
\item Put it together: $r' = r_{\parallel} + (cos \phi) r_{\perp} + (sin \phi)
       \hat{n} \times r$
\end{enumerate}
\end{itemize}
\end{itemize}

Example of something he spent a long time on (note we do a TS expansion of
exponential):

$$e^{\alpha \frac{\partial}{\partial x}}f(x) = (1 + \alpha
\frac{\partial}{\partial x} + \frac{\alpha^2}{2!} \frac{\partial ^2}{\partial
x^2} + \dots )f(x) = f(x) \alpha f'(x) +\frac{\alpha^2}{2!} f''(x) = f(x +
\alpha)$$
\subsubsection{Clifford Algebra $Cl_3$}
\label{sec-2-2-2}


We will define multiplication in this space as $$AB = A \cdot B + i A \times
B$$, with $i = e_1 e_2 e_3$

There are 8 basis elements in a Clifford Space:
\begin{itemize}
\item In $\mathbb{R}: 1$
\item In $i\mathbb{R}: i$
\item In $\mathbb{R}^3: e_1, e_2, e_3$
\item In $i\mathbb{R}^3: ie_1, ie_2, ie_3$
\end{itemize}

These relationships can be summarized in the following table:


\begin{center}
\begin{tabular}{rllll}
 grade  &  domain            &  basis elements                                                                                                                                                                                                                        &  vector types (geometry)    &  who       \\
\hline
     0  &  $\mathbb{R}$      &  1                                                                                                                                                                                                                                     &  scalars                    &  German    \\
     1  &  $\mathbb{R}^2$    &  $\bm{\hat{e_1}}$, $\bm{\hat{e_2}}$,  $\bm{\hat{e_3}}$                                                                                                                                                         &  vectors                    &  French    \\
     2  &  $i \mathbb{R}^3$  &  $i \bm{\hat{e_1}} = \bm{\hat{e_2}} \bm{\hat{e_3}}$,  $i \bm{\hat{e_2}} = \bm{\hat{e_3}} \bm{\hat{e_1}}$,  $i \bm{\hat{e_3}} = \bm{\hat{e_1}} \bm{\hat{e_2}}$  &  bivectors                  &  British   \\
     3  &  $i \mathbb{R}$    &  $i =\bm{\hat{e_1}} \bm{\hat{e_2}} \bm{\hat{e_3}}$                                                                                                                                                             &  trivector (pseudo scalar)  &  Japanese  \\
\end{tabular}
\end{center}



We can decompose the matrix product of a clifford algebra into a symmetric part
and an anti-symmetric part. In other words $$AB = (AB)_{sym} + (AB)_{non-sym} =
A \cdot B + A \wedge B$$
\begin{itemize}

\item Sub-algebras\\
\label{sec-2-2-2-1}%
Sub-algebras are simply subsets of an algebra where all objects are closed
under multiplication and addition. For the clifford algebra, taking scalars and
bivectors we end up with the even sub algebra (made up of grades 0 and 2).


\item More facts\\
\label{sec-2-2-2-2}%
The bivectors define oriented surfaces (oriented because $e_1 e_2 = - (e_2
e_1)$ -- direction matters).

The tri-vectors define an oriented volume.

\end{itemize} % ends low level
\subsubsection{Trig}
\label{sec-2-2-3}

We will need to know certain trig identities. Among them are the following
(Note that bold letters are vectors, lower case letters are magnitude of
vectors and capital letters are angles pointing to legs):
\begin{itemize}
\item Law of cosines: $\bm{c} = \bm{a} - \bm{b}$ and $c^2$ =
  a$^2$ + b$^2$ - 2 \bm{a} $\cdot$ \bm{b}
\item Law of sines: $\bm{c} + \bm{a} + \bm{b} = 0
  \rightarrow |\bm{a} \times \bm{b}| = |\bm{a} \times
  \bm{c}|$. We also know that $ab sin(C) = ac sin(B) \rightarrow
  \frac{sin(B)}{b} = \frac{sin(C)}{c}$
\end{itemize}
\subsubsection{Differential Calculus}
\label{sec-2-2-4}

    I know all this stuff
\subsection{Basic E\&M}
\label{sec-2-3}
\subsubsection{Summary}
\label{sec-2-3-1}

$$\nabla^-1 = \frac{\nabla}{\nabla^2}$$
$$ \frac{1}{\nabla} = g \ast $$
$$ g = \frac{1}{4 \pi r} \text{ Solution to } \nabla^2 g = \delta(r)$$
$$\int_V \nabla \cdot \bm{E} d \tau = \oint_{dV} \bm{E} \cdot d \bm{a} $$
$$\int_S (\nabla \cdot \bm{B}) d \bm{a} = \oint_{S} \bm{B} \cdot d \bm{l} $$
$$ \nabla \bm{E} = \frac{1}{\varepsilon_0} \rho \quad \rho \text{ is charge density}$$
\subsection{Midterm}
\label{sec-2-4}

\begin{itemize}
\item Need to take it on Tuesday the 28th.
\item There will be 4 problems like homework difficulty
\item It is open book and open notes, open old homeworks
\begin{itemize}
\item We will not be allowed to have internet or solutions manual
\end{itemize}
\item We will go over it in class after it is graded
\item Chapters 1-2 and part of 3.
\end{itemize}

\end{document}
