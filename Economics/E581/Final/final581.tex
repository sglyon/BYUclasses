%% Preamble
	\documentclass[paper=a4, fontsize=11pt]{scrartcl} % A4 paper and 11pt font size

	\usepackage[T1]{fontenc} % Use 8-bit encoding that has 256 glyphs
	\usepackage{fourier} % Use the Adobe Utopia font for the document - comment this line to return to the LaTeX default
	\usepackage[english]{babel} % English language/hyphenation
	\usepackage{amsmath,amsfonts,amsthm} % Math packages
	\usepackage{booktabs}
	\usepackage{topcapt}

	\usepackage{lipsum} % Used for inserting dummy 'Lorem ipsum' text into the template

	\usepackage{[pdfsync}
	\usepackage{setspace}
	\usepackage{vmargin}
	\setmarginsrb   {0.70in}  % left margin
	                       {0.70in}   % top margin
	                       {0.70in}   % right margin
	                       {0.70in}  % bottom margin
	                       {  20pt}   % head height
	                       {0.25in}   % head sep
	                       {   9pt}    % foot height
	                       { 0.3in}    % foot sep

	% \usepackage{sectsty} % Allows customizing section commands
	% \allsectionsfont{\centering \normalfont\scshape} % Make all sections centered, the default font and small caps

	\usepackage{fancyhdr} % Custom headers and footers
	\fancyhead[L]{Spencer Lyon: 581 final} % No page header - if you want one, create it in the same way as the footers below
	\fancyhead[R]{} % Empty left footer
	\fancyhead[C]{} % Empty center footer
	\fancyfoot[L]{} % Empty left footer
	\fancyfoot[C]{} % Empty center footer
	\fancyfoot[R]{\thepage} % Page numbering for right footer
	\renewcommand{\headrulewidth}{.5pt} % Remove header underlines
	\renewcommand{\footrulewidth}{.3pt} % Remove footer underlines
	\setlength{\headheight}{13.6pt} % Customize the height of the header

	% \numberwithin{equation}{section} % Number equations within sections (i.e. 1.1, 1.2, 2.1, 2.2 instead of 1, 2, 3, 4)
	\numberwithin{figure}{section} % Number figures within sections (i.e. 1.1, 1.2, 2.1, 2.2 instead of 1, 2, 3, 4)
	\numberwithin{table}{section} % Number tables within sections (i.e. 1.1, 1.2, 2.1, 2.2 instead of 1, 2, 3, 4)

	\setlength\parindent{0pt} % Removes all indentation from paragraphs - comment this line for an assignment with lots of text

	\newcommand{\horrule}[1]{\rule{\linewidth}{#1}} % Create horizontal rule command with 1 argument of height

	\title{
	\normalfont \normalsize
	\textsc{BYU Economics} \\ [25pt] % Your university, school and/or department name(s)
	\horrule{0.5pt} \\[0.4cm] % Thin top horizontal rule
	\huge Econ 581 Final \\ % The assignment title
	\horrule{2pt} \\[0.5cm] % Thick bottom horizontal rule
	}

	\author{Spencer Lyon} % Your name

	\date{\normalsize\today} % Today's date or a custom date

\begin{document}

\maketitle % Print the title
\thispagestyle{empty}
\pagestyle{fancyplain}

%----------------------------------------------------------------------------------------
%	PROBLEM 1
%----------------------------------------------------------------------------------------

\section{The Households' Problem}

	\subsection{Population Growth}

		Let $\lambda_{s, t}$ represent the size of the generation of workers of age $s$ in period $t$. Without loss of generality I make the following normalization: $$\lambda_{S, 0} = 1$$ Looking at all other generations in period zero I get the following:

		\begin{align} \label{eq:lamGen}
		\begin{split}
			\lambda_{S-1, 0} &= (1 + n) \lambda_{S, 0} = (1 + n) \\
					\lambda_{S-2, 0} &= (1 + n) \lambda_{S - 1, 0} = (1 + n) ^ 2 \\
					... \\
					\lambda_{S-s, 0} &= (1 + n)^s
		\end{split}
		\end{align}

		A similar analysis holds for the $S$ aged generation in different periods $t$:

		\begin{align} \label{eq:lamTime}
		\begin{split}
			\lambda_{S, 1} &= \lambda_{S-1, 0} = (1 + n) \\
					\lambda_{S, 2} &= \lambda_{S - 2, 0} = (1 + n) ^ 2 \\
					... \\
					\lambda_{S, t} &= (1 + n)^t
		\end{split}
		\end{align}

		Combining these two results I can an expression for the size of any generation in any time period $$\lambda_{S-s, t} = (1 + n) ^{s + t}$$

	\subsection{Objective Function}

		Each period, households make two decisions: (1) how much of their income to consume in the current period, (2) how much labor to supply. Consumption must be strictly positive and labor is non-negative. If an agent decides to supply 0 labor in a period, they have effectively made the decision to retire and their labor in each subsequent period will also be 0.

		The budget constraint a household of age $s$ in period $t$ faces has the form

		\begin{align} \label{eq:budget}
			c_{s, t} + k_{s+1, t+1} = w_t f_s l_{t, s} (1 - \tau) + (1 + r_t - \delta) k_{s, t}
		\end{align}

		Where

		\begin{itemize}
			\item $c_{s, t}$ is the consumption of a household of age $s$ in period $t$
			\item $k_{s+1, t+1}$ is the savings by an agent currently of age $s$ for the next period ($t + 1$)
			\item $w_t$ is the wage in period $t$
			\item $f_s$ is the productivity of an agent in the $s$th period of his life
			\item $l_{ws t}$ is the labor supplied by the agent
			\item $\tau$ is the tax rate on labor income
			\item $r_t$ is the interest rate of capital
			\item $k_{s, t}$ is what the agent currently of age $s$ saved last period ($t-1$) to consume in period $t$
		\end{itemize}

		Because agents have productivity according to their age only, I will, without loss of generality, represent each generation with a single representative agent. This simplification allows me to conclude that in all periods $t$ there are exactly $S$ households involved in the economy. I make two more assumptions about households and their budget constraints: first that agents begin their lives with no capital ($k_{1, t} = 0 \forall t$) and second that agents consume all their income in their last period of life ($k_{S, t} = 0 \forall t$).

		Agents have a utility function defined in terms of consumption and leisure.

		\begin{align}
			u(c_{s, t}, 1 - l_{s, t}) = \frac{1}{1 - \gamma}(c^{1 -\gamma} - 1) + B ln(1 - l_{s,t})
		\end{align}

%----------------------------------------------------------------------------------------
%	PROBLEM 2
%----------------------------------------------------------------------------------------

\section{The Firm's Problem}

Note about firms.

I believe that I need to include population growth in the production function. The way to do it is to say that $L_t = \sum_{s=1} ^ S \lambda_{s, t} \times l_{s, t}$

\section{The Governments Behavior}

The behavior of the government is just a simple identity. Each period they will collect $T = \sum_{s=1}^S $

\section{Market Clearing Conditions}
	There are three market clearing conditions that must be met in equilibrium

	\begin{itemize}
		\item Capital market $K_t = \sum_{s=2}^S k_{s, t}$
		\item Labor market: $L_t = \sum_{s = 1} ^S l_{s, t}$
		\item Goods market: $Y_t = C_t + K_{t+1} - (1 - \delta)K_t$
	\end{itemize}

	One of these conditions is redundant in equilibrium by Walras' law. Generally, the goods market conditions isn't directly used in solving the model.

\section{Stationarized Equations}

\section{Dynare Equations}

\section{Variables and Parameterization}

List of variables

\begin{itemize}
	\item Endogenous state
	\begin{itemize}
		\item $k_{s+1, t+1} \text{ } \forall s, t$
		\item $l_{s, t} \text{ } \forall s, t$
	\end{itemize}
	\item Exogenous State
	\begin{itemize}
		\item $z_t$
	\end{itemize}
\end{itemize}

Below is a table of parameters and potential values for them

% Requires the booktabs if the memoir class is not being used
\begin{table}[htbp]
   \centering
   % \topcaption{Table captions are better up top} % requires the topcapt package
   \begin{tabular}{@{} llr @{}} % Column formatting, @{} suppresses leading/trailing space
      \toprule
      \multicolumn{2}{c}{Item} \\
      \cmidrule(r){1-2} % Partial rule. (r) trims the line a little bit on the right; (l) & (lr) also possible
      Parameter    & Description & Value\\
      \midrule
			$\alpha$    & Capital share of income & 0.35 \\
			$\delta$     & depreciation rate     &  $1 - (1 - 0.05)^{60/S}$ \\
			$\beta$      & discount factor  & $0.96^{60/S}$ \\
			$\sigma$    & standard deviation of shock process  & $(\sum_{i=1}^{60/S} \rho^{2(\frac{60}{S} - i)}) - \sigma_{ann}$ \\
			$\mu$   	 & mean of shock process  & 0 \\
			$\rho$        & persistence of shock process& $0.95^{4 \frac{60}{S}}$ \\
			$\gamma$  & coefficient of relative risk aversion  & 2 \\
			$B$            & LOOK THIS UP  & ??? \\
			$b$       	 & slope term in $f_s$  & no standard value \\
			$c$      	 & intercept term in $f_s$  & no standard value \\
			$n$ 		 & growth rate of generations  & 0.01 (1\%) \\
      \bottomrule
   \end{tabular}
   \caption{Parameters in the model and potential values}
   \label{tab:booktabs}
\end{table}



\section{Steady State}

\end{document}
